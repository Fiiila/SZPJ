\documentclass{article}

\usepackage{mathptmx}
%packages for language
\usepackage[czech]{babel}
\usepackage[utf8]{inputenc}
%packages for graphic
\usepackage{graphicx}
\usepackage{pgfplots}
\usepackage{pgfplotstable}
\usepackage{multirow}
\usepackage{tikz}
\usepackage{import}
\usepackage[a4paper, total={17cm,25.7cm}, top=2cm, left=2cm, includefoot]{geometry}
\usepackage{todonotes}
\usepackage{standalone}
\usepackage{colortbl}%pro barevne zmeny v tabulce
\usepackage{float}
\usepackage{csvsimple} %pro import a práci s csv soubory
\usepackage{indentfirst}  % odsazení prvního řádku v odstavci
\usepackage{hyperref} %dela odkazy na mista v dokumentu atd
\usepackage{amsmath}%psani matic
\usepackage{mathrsfs}%psani kroucenym matematickym pismem
\usepackage{pdfpages}%vkladani celych pdf dokumentu
\usepackage{listings, lstautogobble}% pro renderovani kodu
%cesta k obrazkum: ./Graphics/....
\lstset{frame=tb, autogobble=true}

\begin{document}
	\begin{titlepage}
    \begin{center}
        \LARGE
        Západočeská Univerzita v Plzni\\
        Fakulta Aplikovaných Věd\\
        
        \vspace{1cm}
        
        \includegraphics[width=0.5\textwidth]{./Graphics/FAV_logo.pdf}
        
        \vspace{4cm}
        
        \textbf{Vektorový model pro vyhledávání informací}
        
        \vspace{0.5cm}
        Filip Jašek
        
    \end{center} 
    \vfill
        \noindent
        \large
        Předmět: SZPJ\\
        Vyučující: Doc. Ing. Ircing Pavel, Ph.D.\hfill Datum: \today
\end{titlepage}
	
	\tableofcontents
	
	\includepdf[page=1, pagecommand=\section{Zadání}]{./SZPJ_SP1_zadani.pdf}
	\includepdf[page=2-, pagecommand={}]{./SZPJ_SP1_zadani.pdf}
	
	\section{Úvod}
		Cílem této semestrální práce bylo implementovat v jazyce Python vektorový model s tf-idf vahami pro vyhledávání informací. Skript byl napsán tak aby zpracovával dotazy zadané v xml souboru a nikoliv interakvivní formou. Čistě pro testování přístupu a jeho benchmark.
		
		Zdrojový kód je dostupný i s návodem k použití na GitHub repository (\href{https://github.com/Fiiila/SZPJ.git}{https://github.com/Fiiila/SZPJ.git}).
		
	\section{Dokumentace skriptu}
		\subsection{Předzpracování}
			Při prvním kroku implementace bylo potřeba správně naparsovat zdrojové soubory html a identifikovat tak jen relevantní obsah. Jako první byly vymazány řádky obsahující tagy například \verb*|<html>,</html>,<pre>,</pre>|. Dále pak byly identifikovány pro vyhledávání nerelevantní řádky obsahující pouze číselné hodnoty a také řádek s informací pravděpodobně o přístupu ke článku ve tvaru například \textit{CA581203 JB March 22, 1978  8:28 PM}.
			
			Po vynechání zmíněných řádků byly zbývající zbaveny prázdných charakterů na okrajích funkcí \verb*|strip()| a spojeny do skupin dle rozdělení prázdnými řádky. Tato funkcionalita je sice nadbytečná ale v případě vyhledávání například ve specifickém roce nebo v jedné části textu by mohla zjednodušit a zrychlit vyhledávání zejména při použití většího zdrojového datasetu za pomoci filtrace nebo vytváření subsetů pro vektorizaci.
			
			Při samotném vytvoření datasetu byla navíc odstraněna základní interpunkce, aby se eliminovala souvislost hledaných výrazů a interpunkce. V tomto duchu byla později implementovaná i metoda stemming z knihovny \href{https://www.nltk.org/howto/stem.html}{NLTK}\footnote{https://www.nltk.org/howto/stem.html}, která v nejrůznějších tvarech slov nalezne kořen slova čímž je pak původní slovo v dokumentu nahrazeno. Jednotlivá slova byla rozdělena tokenizací pomocí knihovnou \href{https://www.nltk.org/api/nltk.tokenize.word_tokenize.html}{NLTK}\footnote{https://www.nltk.org/api/nltk.tokenize.word\_tokenize.html} Stejný postup se použije i při načítání jednotlivých dotazů pro vyhledávání. Celkové MAP bylo tak zlepšeno o \(\approx10.62\%\).
			
		\subsection{Vektorový model}
			Zvolený vektorový model byl z knihovny \href{https://scikit-learn.org/stable/modules/generated/sklearn.feature_extraction.text.TfidfVectorizer.html}{skicit-learn}\footnote{https://scikit-learn.org/stable/modules/generated/sklearn.feature\_extraction.text.TfidfVectorizer.html} prezentovaný na přednáškách pro jeho jednoduchost. Při experimentaci s parametry bylo dosaženo nejlepších výsledků s následujícím nastavením.
			
			\begin{lstlisting}[language=Python]
				# Specify vectorizer object and transform data
				tfidf = TfidfVectorizer(norm=None, use_idf=True, smooth_idf=True, 
										sublinear_tf=True, 
										stop_words='english'
										)
			\end{lstlisting}
			
			Normalizace výstupních řádků nepřinesla žádné zlepšení, naopak \verb*|use_idf| pomohlo z podstaty věci aktivovat kdy při deaktivaci je \(idf(t)=1\). Vyhlazení vážení idf pomocí \verb*|smooth_idf| také zlepšilo MAP podobně jako použití logaritmického škálování \verb*|tf| jako \(1+\log(tf)\). Výhoda byla také při použití parametru \verb*|stop_words='english'| která při tokenizaci eliminuje slova které nemají velký vliv na samotnou informaci obsaženou v textu jako je 'the', 'and', 'him'.
			
	\newpage		
	\section{Závěr}
		V této semestrální práci byly aplikovány znalosti z přednášek předmětu SZPJ, zejména pro rozdělení informace na tokeny a samotné vektorové vyhledávání. Testováním a použitím všech zmíněných argumentů a metod při parsování a výběru podstatných informací bylo dosaženo na testovacím datasetu skóre \(MAP\approx0.3625\).
			
			
			
			
		
	
\end{document}
